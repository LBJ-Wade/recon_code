\documentclass[a4paper,11pt]{article}
\pdfoutput=1 % if your are submitting a pdflatex (i.e. if you have
             % images in pdf, png or jpg format)

\usepackage{jcappub} % for details on the use of the package, please
                     % see the JCAP-author-manual

\usepackage[T1]{fontenc} % if needed

\usepackage{amssymb, amsmath, epsfig, natbib}
\def\apj{ApJ}
\def\apjl{ApJL}
\def\apjs{ApJS}
\def\aj{AJ}
\def\nat{Nature}
\def\mnras{MNRAS}
\def\physrep{physrep}
\def\pasj{PASJ}
\def\araa{{Ann.\ Rev.\ Astron.\& Astrophys.\ }}
\def\aap{{A\&A}}
\def\prd{PRD}
\def\jcap{JCAP}

\author[a,b]{Martin White}

\affiliation[a]{Department of Physics, University of California,
Berkeley, CA}
\affiliation[b]{Lawrence Berkeley Laboratory, 1 Cyclotron Road,
Berkeley, CA}

\emailAdd{mjwhite@lbl.gov}


\title{Reconstruction Code}

\keywords{cosmological parameters from LSS -- baryon acoustic oscillations
-- galaxy clustering}
%\maketitle

\abstract{
We present a (C++) code to perform reconstruction on a galaxy catalog, to
sharpen the baryon acoustic oscillation feature for measuring distances.
The code implements the standard, lowest-order algorithm presented in
\cite{Eis07} with periodic boundary conditions using a multigrid relaxation
technique with a full multigrid V-cycle based on damped Jacobi iteration.
Inputs are a catalog of objects and two random catalogs which serve to
specify the selection function and act as the "shifted" fields. Output
are `shifted' versions of the object and second random catalogs.
}

\begin{document}
\maketitle
\flushbottom

\section{Introduction}
\label{sec:intro}

Sound waves traveling in the early Universe imprint a characteristic
scale in the large-scale clustering of objects which can be used as a
standard ruler to probe the expansion history of the Universe.  These
so-called baryon acoustic oscillations are partially damped by
non-linear evolution, and techniques to remove this damping are now
routinely applied in the analysis of survey data and assumed in
forecasts of future performance.  This process of `reconstruction'
comes in several flavors, but the most widely used is that of
\cite{Eis07} (which itself draws on earlier work going back decades).
In this note we present a new implementation of this `standard'
reconstruction code which uses an efficient multigrid solver to
compute the displacements along which objects are moved in the process
of reconstruction.

There has been an extensive study of this `standard' reconstruction procedure
in the literature, from both the analytic \cite{PWC09,NWP09,TasZal12,ZelRecon}
and numerical \cite{Seo10,Pad12,Xu13,Bur14,Toj14,Var14,Bur15} perspectives.
These works discuss the robustness of the algorithm, its performance and
several issues of implementation.
A particularly detailed discussion in given in \cite{Pad12,Bur14,Bur15}.
The only new piece of the current code is the use of a multigrid method to
solve for the displacement, $\vec{\Psi}$, given the density field.  Earlier
codes used Fourier transform methods or linear algebra libraries.  Multigrid
techniques provide an efficient alternative which can easily handle
non-parallel lines of sight, require no external libraries and allow very
efficient parallelism with multiple threads.

\section{Procedure}

The inputs are three files (containing RA, DEC, redshift and possibly a weight)
for the objects and two random catalogs, a linear bias factor ($b$), a growth
parameter ($f$) and a Gaussian filtering scale ($R_f$).
The first random catalog is used to define the selection function of the
survey (and hence to compute the density field).  This should be large.  The
second random catalog is used to generate the `shifted' field which is needed
for computation of the post-reconstruction clustering signal.
The files are read and angular coordinates are converted to 3D Cartesian
coordinates using a $\Lambda$CDM distance-redshift relation.  The default
value is $\Omega_m=0.3$ (set in {\tt recon.cpp}) and distances are computed
in $h^{-1}$Mpc so that no value of $h$ needs to be specified.  The distance
calculation includes only matter and a cosmological constant, there is no
contribution from radiation or massive neutrinos.
The outputs are the Cartesian coordinates of the `shifted' object and
random catalogs, with the shift vectors computed using lowest order
Lagrangian perturbation theory with the assumed (linear) bias, $b$, and
rate-of-growth parameter, $f$.

The code first estimates densities by depositing the objects and the first
set of randoms onto a regular Cartesian mesh using CIC interpolation in a box
which encloses the random catalog entirely with a 25 per cent margin (for
current and future surveys this is likely sufficient).
The grid size is defined by the variable {\tt Ng} in {\tt global.h}.  It
should be set such that the grid spacing is slightly smaller than the smoothing
length [it would not be difficult to modify the code to do this automatically,
but currently it is assumed that the physical set-up will not change
frequently].  In principle any value of {\tt Ng} can be used but the code is
more efficient if the prime factors are few and low numbers.  Powers of two or
three times powers of two work well.  Significantly different choices may
require retuning of the parameters in the multigrid algorithm.

A grid of $\delta=\rho_{\rm obj}/\rho_{\rm ran}-1$ is formed and regions
with $\rho_{\rm ran}=0$ are set to $\delta=0$ ensuring padding at the
mean density.
At this stage the density constrast is divided
by the (supplied) large-scale bias, $b$, to convert from object fluctuations
to mass fluctuations.
Note: as with most situations involving division by a gridded density field,
it is important that there be enough randoms that the density field on the
grid is well defined!  To help mitigate numerical errors associated with this
problem, any grid point with too few randoms is assigned the mean density.
The density contrast is then Gaussian smoothed by brute-force\footnote{Aside
{}from the ascii I/O, this is the slowest step in the code -- for this reason
we have also added an option to smooth the density field using FFTs if you
have the FFTW library (v3).} convolution with a kernel $\exp[-(x/R_f)^2/2]$.

We solve for the Zeldovich \cite{Zel70} displacements, $\vec{\Psi}$, using
a standard multigrid technique employing a full multigrid V-cycle with
damped Jacobi iteration (see below).
Both the data and the randoms are then moved by $-\vec{\Psi}$ to form the
so-called `displaced' and `shifted' fields.
We choose to move both the objects and the randoms by an additional factor
of $f\hat{r}\hat{r}\cdot\vec{\Psi}$ (i.e.~in the line-of-sight direction).
An alternative, and common, choice is to enhance the shift in the line-of-sight
direction only for the objects, and not the randoms
(see \cite{ZelRecon} for further discussion and comparison).
The Cartesian coordinates, with $h^{-1}$Mpc units, for the shifted and
displaced fields (along with any weights) are then written to file.

Currently all of the I/O is done with ascii flat files.  This can be time
consuming, especially for large catalogs.  A more efficient implementation
would read and write binary files of some form.

\section{Solving for $\vec{\Psi}$}
\label{sec:equations}

We need to extract the displacement field, $\vec{\Psi}$, from the observed,
redshift-space density field.  In this section we present the formalism for
an unbiased tracer of the density field -- for biased tracers the source,
$\delta$, and the factor $f$ both\footnote{Note, in the literature this change
of $f\to\beta=f/b$ is sometimes misprinted.  The easiest way to derive the need
for this is to consider a plane-parallel situation with $\hat{r}=\hat{z}$ and
linear theory redshift space distortions in $k$-space:
$\delta_k\to (b+f\mu_k^2)\delta_k$ and look at Eq.~(\ref{eqn:pde}).}
need to be divided by $b$.

Within the Zeldovich approximation \cite{Zel70}
\begin{equation}
  \vec{\nabla}\cdot\vec{\Psi} + f\vec{\nabla}\cdot
  \left(\hat{r}\left\{\hat{r}\cdot\vec{\Psi}\right\}\right)
  = -\delta
\end{equation}
To first order $\vec{\Psi}$ is irrotational, so we can write
$\vec{\Psi}=\nabla\phi$ and hence
\begin{equation}
  \nabla^2\phi +
  f\vec{\nabla}\cdot\left( \left[\hat{r}\cdot\vec{\nabla}\phi\right]
  \hat{r}\right) = -\delta
\label{eqn:pde}
\end{equation}
To simplify the equations slightly note that we can rewrite the term
linear in $f$ as
\begin{equation}
 \vec{\nabla}\cdot\left( \left[\hat{r}\cdot\vec{\nabla}\phi\right]\hat{r}\right)
 = \sum_{ij}\hat{r}_i\hat{r}_j\left(\partial_i\partial_j\phi\right)
  +  \frac{2}{r}\hat{r}_j\partial_j\phi
\label{eqn:fterm}
\end{equation}

Discretizing the second derivatives to second order\footnote{We use a central
difference scheme, which gives second-order accurate approximations.  These
formulae are most easily derived by Taylor series expansion.} 
we have, e.g.
\begin{equation}
  \partial_x^2\phi \approx \frac{\phi_{i-1,j,k}-2\phi_{i,j,k}+\phi_{i+1,j,k}}
  {h^2}
\end{equation}
where $h$ is the mesh size.  We shall write this second-order accurate
difference as $D_x^2$.  This leads to a 3D stencil for $\nabla^2$ with
six $1$'s along the Cartesian directions and $-6$ in the center.

If we approximate the radial direction appearing in Eq.~(\ref{eqn:pde})
as that pointing to the central mesh point (i.e.~again neglect the variation
of the line-of-sight direction in the derivative) then the term proportional
to $f$ in the above becomes simply
\begin{equation}
  \sum_{i,j=1}^3 \hat{r}_i\hat{r}_j D_iD_j \phi
  + \frac{2}{r}\sum_{j=1}^3 \hat{r}_jD_j\phi
\label{eqn:ffinite}
\end{equation}
Consider the second derivative term.
The $i=j$ components of the central difference all involved $\phi_{i,j,k}$
while the others will not.  Since $\hat{r}\cdot\hat{r}=1$ the coefficient
of the central, $\phi_{i,j,k}$ term is thus replaced by $-6\to -6(1+f/3)$
while the coefficients of the terms offset by $\pm 1$ become slightly more
complex.  If we use a central difference for the $1^{\rm st}$ derivative,
it is also independent of $\phi_{i,j,k}$.

Continuing to work to $2^{\rm nd}$ order the central differences appear as, e.g.
\begin{equation}
  \partial_x\partial_y\phi \approx
  \frac{\phi_{i+1,j+1,k}-\phi_{i+1,j-1,k}-\phi_{i-1,j+1,k}+\phi_{i-1,j-1,k}}
       {4h^2}
\end{equation}
and
\begin{equation}
  \partial_x\phi \approx \frac{\phi_{i+1,j,k}-\phi_{i-1,j,k}}{2h}
\end{equation}
which do not include $\phi_{i,j,k}$.
Eq.~(\ref{eqn:ffinite}), times $h^2$, then becomes
\begin{eqnarray}
  &-& 2\phi_{i,j,k} \nonumber \\
  &+&\hat{r}_x^2 \left( \phi_{i-1,j,k}+\phi_{i+1,j,k} \right)
   + \hat{r}_y^2 \left( \phi_{i,j-1,k}+\phi_{i,j+1,k} \right)
   + \hat{r}_z^2 \left( \phi_{i,j,k-1}+\phi_{i,j,k+1} \right)   \nonumber \\
  &+&\hat{r}_x\hat{r}_y\left( \phi_{i-1,j-1,k}+\phi_{i+1,j+1,k}
                             -\phi_{i+1,j-1,k}-\phi_{i-1,j+1,k} \right)/2
      \nonumber \\
  &+&\hat{r}_x\hat{r}_z\left( \phi_{i-1,j,k-1}+\phi_{i+1,j,k+1}
                             -\phi_{i+1,j,k-1}-\phi_{i-1,j,k+1} \right)/2
      \nonumber \\
  &+&\hat{r}_y\hat{r}_z\left( \phi_{i,j-1,k-1}+\phi_{i,j+1,k+1}
                             -\phi_{i,j+1,k-1}-\phi_{i,j-1,k+1} \right)/2
      \nonumber \\
  &+&\theta\hat{r}_x\left(\phi_{i+1,j,k}-\phi_{i-1,j,k}\right)
   + \theta\hat{r}_y\left(\phi_{i,j+1,k}-\phi_{i,j-1,k}\right)
   + \theta\hat{r}_z\left(\phi_{i,j,k+1}-\phi_{i,j,k-1}\right)
\end{eqnarray}
where we have written $\theta=h/r$ and in practice inclusion of these
$\theta$-terms makes little difference to most problems.
Note, if we take the line-of-sight direction to be fixed throughout the grid,
e.g.~$\hat{r}\simeq\hat{z}$ and take $\theta=0$, then $\hat{r}_x=\hat{r}_y=0$
and $\hat{r}_z=1$ so the sum of the $\nabla^2$ and $f$ terms reduces to
increasing the $D_z^2$ central difference by a factor of $1+f$ while leaving
all the others unchanged, as expected.

Having discretized the equations, we can now rewrite them in a form
conducive to relaxation
\begin{eqnarray}
  (6+2f)\ \phi_{i,j,k} &=& h^2 \delta_{i,j,k} \nonumber \\
  &+&(1+f\hat{r}_x^2)\left( \phi_{i-1,j,k}+\phi_{i+1,j,k} \right) \nonumber \\
  &+&(1+f\hat{r}_y^2)\left( \phi_{i,j-1,k}+\phi_{i,j+1,k} \right) \nonumber \\
  &+&(1+f\hat{r}_z^2)\left( \phi_{i,j,k-1}+\phi_{i,j,k+1} \right) \nonumber \\
  &+&f\theta\hat{r}_x\left(\phi_{i+1,j,k}-\phi_{i-1,j,k}\right)
   + f\theta\hat{r}_y\left(\phi_{i,j+1,k}-\phi_{i,j-1,k}\right)
   + f\theta\hat{r}_z\left(\phi_{i,j,k+1}-\phi_{i,j,k-1}\right) \nonumber \\
  &+&f\hat{r}_x\hat{r}_y\left( \phi_{i-1,j-1,k}+\phi_{i+1,j+1,k}
                              -\phi_{i+1,j-1,k}-\phi_{i-1,j+1,k} \right)/2
      \nonumber \\
  &+&f\hat{r}_x\hat{r}_z\left( \phi_{i-1,j,k-1}+\phi_{i+1,j,k+1}
                              -\phi_{i+1,j,k-1}-\phi_{i-1,j,k+1} \right)/2
      \nonumber \\
  &+&f\hat{r}_y\hat{r}_z\left( \phi_{i,j-1,k-1}+\phi_{i,j+1,k+1}
                              -\phi_{i,j+1,k-1}-\phi_{i,j-1,k+1} \right)/2
\end{eqnarray}
We solve this system using a very standard multigrid technique, using a full
multigrid with a V-cycle based upon damped Jacobi iteration.
We implemented the algorithm using the simplest V-cycle.
The number of V-cycle calls is defined by the {\tt Niter} parameter in
the routine {\tt fmg} in {\tt multigrid.cpp} and the number of Jacobi
steps per level is set by the {\tt Nit} parameter in {\tt jacobi} in
{\tt multigrid.cpp}.
Intermesh transfers are done using a standard stencil, with full-weighting
for the restriction operator.
The code is {\tt OpenMP} parallel, assuming shared memory.

Having obtained $\phi$ on the grid we compute $\vec{\Psi}$ using the
second-order, central difference approximation as above, e.g.
\begin{equation}
  \Psi_x = \nabla_x\phi \approx \frac{\phi_{i+1,j,k}-\phi_{i-1,j,k}}{2h}
\end{equation}
and similarly for $\nabla_y$ and $\nabla_z$.  The shift vectors at the
positions of the objects and randoms are then obtained from CIC interpolation
and the positions moved by $\vec{r}\to\vec{r}-\mathbf{R}\vec{\Psi}$ where
$R_{ij}=\delta_{ij}+f\hat{r}_i\hat{r}_j$ is a redshift-space operator.

\bibliographystyle{JHEP}
\bibliography{notes}
\end{document}
